%%%%%%%%%%%%%%%%%%%%%%%%%%%%%%%%%%%%%%%%%
% Medium Length Professional CV
% LaTeX Template
% Version 2.0 (8/5/13)
%
% This template has been downloaded from:
% http://www.LaTeXTemplates.com
%
% Original author:
% Rishi Shah 
%
% Important note:
% This template requires the resume.cls file to be in the same directory as the
% .tex file. The resume.cls file provides the resume style used for structuring the
% document.
%
%%%%%%%%%%%%%%%%%%%%%%%%%%%%%%%%%%%%%%%%%

\documentclass[french]{resume} % Use the custom resume.cls style

\usepackage[utf8]{inputenc}
\usepackage[T1]{fontenc}
\usepackage[dvipsnames]{xcolor}
\usepackage{babel}
\usepackage[symbol]{footmisc}
\usepackage{hyperref} 
\usepackage[left=0.75in,top=0.6in,right=0.75in,bottom=0.6in]{geometry} 

\renewcommand{\thefootnote}{\fnsymbol{footnote}}
\newcommand{\tab}[1]{\hspace{.2667\textwidth}\rlap{#1}}
\newcommand{\itab}[1]{\hspace{0em}\rlap{#1}}
\makeatletter
\newcommand\footnoteref[1]{\protected@xdef\@thefnmark{\ref{#1}}\@footnotemark}
\makeatother
\newcommand{\TODO}[1]{\textcolor{red}{TODO #1}}

\name{Curriculum Vit\ae} 
\address{} 

\begin{document}
	
	\begin{minipage}[t]{0.49\textwidth}
		\begin{tabular}{ m{7em} m{15em} }
			\textbf{Name:} & Jason Schoeters\\
%			\textbf{Nationality:} & Belgian\\
%			\textbf{Year of birth:} & 1993\\
			%\\ [1em]\\[-2.9em]
			\textbf{Website:} & \url{https://jschoete.github.io}\\
			%\\ [1em]\\[-2.9em]
			\textbf{E-mail: } & \href{mailto:jason.schoeters.cs@gmail.com}{jason.schoeters.cs@gmail.com}\\
			\textbf{Phone : } & +44 7472488591\\
		\end{tabular}
	\end{minipage} 
	\begin{minipage}[t]{0.49\textwidth}
		\begin{tabular}{ m{8em} m{20em} }
			\textbf{Office:} & DISIA office 61,\\
			&Florence 50134, \newline Italy\\
			\textbf{Last updated:} & December 4 2024
		\end{tabular}
	\end{minipage} 
	
	\begin{rSection}{Experience}
		{\bf Postdoctoral research fellow} \hfill {\em 2024 - 2025} 
		\\ University of Florence, Italy
		%	\\ $| \qquad$ \textit{Subject:} Extending graph theory concepts in a temporal setting
		\\ $| \qquad$ \textit{Subjects}: Cycles and DLT/blockchains in temporal graphs
		\\ $| \qquad$ at DISIA, collaborating with Andrea Marino 
		\\
		\\{\bf Research associate} \hfill {\em 2023 - 2024} 
		\\ University of Cambridge, United Kingdom
		%	\\ $| \qquad$ \textit{Subject:} Extending graph theory concepts in a temporal setting
		\\ $| \qquad$ \textit{Subjects}: Behavioural complexity in humans and artificial intelligence
		\\ $| \qquad$ at Faculty of Economics, collaborating with Peter Bossaerts
		\\
		\\ {\bf Postdoctoral research fellow} \hfill {\em 2021 - 2022} 
		\\ University of Le Havre, France
		%	\\ $| \qquad$ \textit{Subject:} Extending graph theory concepts in a temporal setting
		\\ $| \qquad$ \textit{Subjects}: Components and dense spanners in temporal graphs
		\\ $| \qquad$ at LITIS, collaborating with Eric Sanlaville
		\\
		\\ {\bf Research and teaching assistant} \hfill {\em 2020 - 2021} 
		\\ University of Bordeaux, France
		\\ $| \qquad$ \textit{Subjects}: Structural and algorithmic geometrical problems
		\\ $| \qquad$ at LaBRI
%		\\ 
%		\\
%		{\bf Visiting graduate student} \hfill {\em winter 2020}
%		\\ Simon Fraser University, Vancouver, Canada
%		\\ $| \qquad$ \textit{Subjects:} Gossiping, Influence diffusion, Temporal spanners
%		\\ $| \qquad$ at School of computing, invited by Joseph G. Peters
	\end{rSection}
	
	\begin{rSection}{Diplomas}
		{\bf French qualification for associate professor} \hfill {\em 2023}
		\\ Higher education and research ministry, France 
%		\\ $| \qquad$ \textit{Subjects:} Gossiping, Influence diffusion, Temporal spanners
%		\\ $| \qquad$ at School of computing, invited by Joseph G. Peters
		\\ \\
%		{\bf Visiting graduate student} \hfill {\em winter 2020}
%		\\ Simon Fraser University, Vancouver, Canada
%		\\ $| \qquad$ \textit{Subjects:} Gossiping, Influence diffusion, Temporal spanners
%		\\ $| \qquad$ at School of computing, invited by Joseph G. Peters
%		\\ \\
		{\bf PhD in Computer Science} \hfill {\em 2017 - 2020} 
		\\ École doctorale Mathématiques et Informatique, Bordeaux, France
		\\ $| \qquad$ \textit{Subject}: Contributions to temporal graph theory and mobility-related problems
		\\ $| \qquad$ at LaBRI, supervised by Arnaud Casteigts
		\\ $| \qquad$ \textit{Research visit}: Simon Fraser University, Vancouver, Canada \hfill {\em winter 2020}
		\\ $| \qquad \qquad$ Gossiping and influence diffusion, invited by Joseph G. Peters 
		\\ \\
		{\bf Master of Theoretical Computer Science} \hfill {\em 2015 -  2017} 
		\\ Collège Sciences et technologies, Université de Bordeaux, France 
		\\ $| \qquad$ \textit{Internship}: VectorTSP\hfill {\em summer 2017} 
		\\ $| \qquad $ at LaBRI, supervised by Arnaud Casteigts
		\\ \\ 
		\textbf{Bachelor of Computer Science}\hfill {\em 2012 -  2015} 
		\\ Collège Sciences et technologies, Université de Bordeaux, France 
		\\ $| \qquad$ \textit{Internship}: Image processing, network theory and graphical art\hfill {\em summer 2013} 
		\\ $|  \qquad $ at LaBRI, supervised by Guy Melançon
	\end{rSection}
	\newpage
	\begin{rSection}{Publications }
		%{\bf Contributions to temporal graph theory and mobility-related problems}\\
		%$| \qquad$ J. Schoeters\\
		%$| \qquad$ TODO \hfill {\em 2020}\\
		%$| \qquad$ \textbf{TODO} 
		%\\ \\
		{\bf Temporal Cycle Detection and Acyclic Temporization}\\
		$| \qquad$ J. Araujo, D. de Andrade, A. Ibiapina, A. Marino, J. Schoeters, A. Silva\\
%		$| \qquad$ Journal version TBD \hfill {\em 2025+}\\
		$| \qquad$ Conference version submitted \hfill {\em 2025+}
		%		$| \qquad$ 16th Int. Symposium on Algorithms and Experiments for Wireless Sensor Networks \hfill {\em 2020}
		%\\
		%$| \qquad$ \textbf{ALGOSENSORS} 
		\\
		\\
		{\bf On inefficiently connecting temporal networks}\\
		$| \qquad$ E. Christiann, E. Sanlaville, J. Schoeters\\
		$| \qquad$ Journal version TBD \hfill {\em 2025+}\\
		$| \qquad$ 3rd Symposium on Algorithmic Foundations of Dynamic Networks (SAND)\hfill {\em 2024}
		%		$| \qquad$ 16th Int. Symposium on Algorithms and Experiments for Wireless Sensor Networks \hfill {\em 2020}
		%\\
		%$| \qquad$ \textbf{ALGOSENSORS} 
		\\
		\\
		{\bf Temporally connected components}\\
		$| \qquad$ S. Balev, E. Sanlaville, J. Schoeters\\
		$| \qquad$ Theoretical Computer Science (TCS)\hfill {\em 2024}
%		$| \qquad$ 16th Int. Symposium on Algorithms and Experiments for Wireless Sensor Networks \hfill {\em 2020}
		%\\
		%$| \qquad$ \textbf{ALGOSENSORS} 
		\\
		\\
		{\bf VectorTSP: A Traveling Salesperson Problem with Racetrack-like acceleration constraints}\\
		$| \qquad$ A. Casteigts, M. Raffinot, J. Schoeters\\
		$| \qquad$ Under revision for Discrete Applied Mathematics (DAM)\hfill {\em 2024+}\\
		$| \qquad$ {\small 16th Int. Symposium on Algorithms and Experiments for Wireless Sensor Networks (IWOCA)} \hfill {\em 2020}
		%\\
		%$| \qquad$ \textbf{ALGOSENSORS} 
		\\
		\\
		{\bf Temporal Cliques Admit Sparse Spanners}\\
		$| \qquad$ A. Casteigts, J.G. Peters, J. Schoeters\\
		$| \qquad$ Journal of Computer Systems and Science, Elsevier (JCSS), Vol. 121, 1-17 \hfill {\em 2021}\\
		$| \qquad$ $46^{th}$ Int. Colloquium on Automata, Languages, and Programming (ICALP) \hfill {\em 2019}
		%\\
		%$| \qquad$ \textbf{ICALP}
	\end{rSection}
	
	\begin{rSection}{Software Development 
			%		\hfill \normalfont \lowercase{(available on github)}
		}
		
		{\bf VectorTSP competition} \hfill {\em 2021}\\
		$| \qquad$ Java program computing VectorTSP benchmarks with multiPointAStar algorithm\\
		$| \qquad$ available on \url{https://github.com/jschoete/competitionVectorTSP}\\
		\\	
		{\bf Estimation, approximation and exact computation of overlapping canopied areas} \hfill {\em 2021}\\
		$| \qquad$ Java program computing canopied areas covered by given buffer zone\\
		$| \qquad$ with Clément Larue\\
		$| \qquad$ available on \url{https://github.com/jschoete/CanopyAreaComputer}\\
		\\	
		{\bf Mobility models inducing temporal graph properties} \hfill {\em 2021}\\
		$| \qquad$ Java library using JBotSim for inducing temporal graph properties in MANET\\
		$| \qquad$ with Arnaud Casteigts\\
		$| \qquad$ available on \url{https://github.com/jschoete/mobilitymodels}\\
		\\
		{\bf Automatic analysis of large DNA genotyping data} \hfill {\em 2020}\\
		$| \qquad$ Java program analyzing Excel data files for DNA parent/child mismatches\\
		$| \qquad$ with Clément Larue\\
		$| \qquad$ available on \url{https://github.com/jschoete/mismatchfinder}
	\end{rSection}
	\newpage
	\begin{rSection}{Talks}
		{\bf Temporal Cycle Detection and Acyclic Temporization} \\
		$| \qquad$ \textit{NESTID seminar}, Durham, United Kingdom\hfill {\em November 15 2024} 
		\\ \\
		{\bf Learning-based classification and generation of temporal cliques} \\
		$| \qquad$ \textit{LIPNE complexity seminar}, Cambridge, United Kingdom\hfill {\em April 12 2024} 
		\\ \\
		{\bf Knapsack Solution Robustness} \\
		$| \qquad$ \textit{LIPNE complexity seminar}, Cambridge, United Kingdom\hfill {\em February 16 2024} 
		\\ \\
		{\bf On inefficiently connecting temporal networks} \\
		$| \qquad$ \textit{TEMPOGRAL workshop}, Honfleur, France \hfill {\em February 7 2024} \\
		$| \qquad$ \textit{Economic networks seminar}, Cambridge, United Kingdom \hfill {\em December 1 2023} \\
		$| \qquad$ \textit{LIPNE complexity seminar}, Cambridge, United Kingdom \hfill {\em October 6 2023} \\
		$| \qquad$ \textit{ICALP temporal graph workshop}, Paderborn, Germany \hfill {\em July 10 2023}
		\\ \\
		{\bf Temporal graph theory: structure and algorithmics} \\
		$| \qquad$ \textit{Microeconomics seminar}, Cambridge, United Kingdom \hfill {\em March 15 2023} 
		\\ \\
		{\bf Temporally connected components} \\
		$| \qquad$ \textit{NESTID seminar}, Durham, United Kingdom\hfill {\em May 4 2023} \\
		$| \qquad$ \textit{AlgoDist seminar}, Bordeaux, France \hfill {\em April 24 2023} \\
		$| \qquad$ \textit{TEMPOGRAL seminar}, Poitiers, France \hfill {\em November 24 2022} \\
		$| \qquad$ \textit{Journées Graphes et Algorithmes}, Paris, France \hfill {\em November 17 2022}
		\\ \\
		{\bf Estimation, approximation and exact computation of overlapping canopied areas} \\
		$| \qquad$ \textit{Heudiasyc CID seminar}, Compiegne, France \hfill {\em April 12 2022} \\
		$| \qquad$ \textit{INRAE Biogeco seminar}, Bordeaux, France \hfill {\em December 10 2021} 
		\\ \\
%		{\bf Notes on dense spanners} \\
%		$| \qquad$ \textit{Temporal graphs seminar}, Dagstuhl, Germany (online)\hfill {\em April 28 2021} 
%		\\ \\	
		{\bf Contributions to temporal graph theory and mobility-related problems} \\
		$| \qquad$ \textit{LaBRI PhD defense}, Bordeaux, France\hfill {\em 
			March 29, 2021} 
		\\ \\	
		%{\bf The Traveling Salesman Problem} \\
		%$| \qquad$ \textit{PubhD vulgarization}, Bordeaux, France\hfill {\em delayed due to COVID19, 2020+} 
		%\\ \\
		{\bf VectorTSP: A Traveling Salesperson Problem with Racetrack-like acceleration constraints}\\
		$| \qquad$ \textit{CITI CHROMA seminar}, Lyon, France\hfill {\em May 10, 2022}\\
		$| \qquad$ \textit{AlgoTel}, La Rochelle, France\hfill {\em September 22, 2021}\\
		$| \qquad$ \textit{LITIS RI2C seminar}, Le Havre, France\hfill {\em June 15, 2021}\\
		$| \qquad$ \textit{TU Berlin Algorithmics Colloquium}, Berlin, Germany (online)\hfill {\em December 8, 2020}\\
		$| \qquad$ \textit{LaBRI distributed algorithms seminar}, Bordeaux, France\hfill {\em September 14, 2020}\\
		$| \qquad$ \textit{ALGOSENSORS}, Pisa, Italy (online) \hfill {\em September 10, 2020} \\
		$| \qquad$ \textit{SFU Theory Seminar}, Vancouver, Canada\hfill {\em March 2, 2020}
%		\\ \\
%		{\bf Racetrack and VectorTSP} \\
%		$| \qquad$ \textit{SFU Theory Seminar}, Vancouver, Canada\hfill {\em March 2, 2020}
		\\ \\
		{\bf Temporal Cliques Admit Sparse Spanners} \\
		$| \qquad$ \textit{LITIS RI2C seminar}, Le Havre, France\hfill {\em May 31, 2022} \\
		$| \qquad$ \textit{ROADEF}, Lyon, France\hfill {\em February 24, 2022} \\
		$| \qquad$ \textit{LIP6 complex networks seminar}, Paris, France\hfill {\em November 10, 2020} \\
		$| \qquad$ \textit{SFU Discrete Maths Seminar}, Vancouver, Canada\hfill {\em February 18, 2020} \\
		$| \qquad$ \textit{AlgoTel}, Narbonne, France (\textbf{best student paper award})\hfill {\em June 4 - 7, 2019} \\
		$| \qquad$ \textit{Workshop CoA}, Roscoff, France\hfill {\em April 3 - 5, 2019} \\
		$| \qquad$ \textit{LaBRI distributed algorithms and graphs seminar}, Bordeaux, France\hfill {\em March 11, 2019} \\
		$| \qquad$ \textit{Journées Graphes et Algorithmes}, Grenoble, France\hfill {\em November 14 - 16, 2018}
	\end{rSection}
	
	\begin{rSection}{Other
		}
		
		{\bf {\small Sexual interference revealed by joint study of male and female pollination success in chestnut}}\\
		$| \qquad$ C. Larue, E. Klein, R. Petit\\
		$| \qquad$ Molecular Ecology \hfill {\em 2022}\\
		$| \qquad$ (Contribution through large DNA genotyping data analysis program)
%		\\
%		\\
%		{\bf The number of labels per edge maintaining temporal connectivity}\\
%		$| \qquad$ J. Schoeters\\
%		$| \qquad$ Dagstuhl seminar report of Temporal Graphs: Structure, Algorithms, Applications\hfill {\em 2021}\\
%		$| \qquad$ (Open problem session)
	\end{rSection}
	
	
	\begin{rSection}{Students}
		
%		{\bf Ahmed Reza Khaen} (Undergraduate IIT Kharagpur) \hfill {\em summer 2024}
%		\\
%		$| \qquad$ \textit{Project}: Learning-based classification and generation of temporal cliques\\
%		$| \qquad$ in collaboration with Nitin Yadav
%		, co-supervised with Eric Sanlaville
%		\\ 
%		\\	
		{\bf Esteban Christiann} (L3 ENS Paris-Saclay) \hfill {\em summer 2022}
		\\
		$| \qquad$ \textit{Internship}: Dense spanners and related problems\\
		$| \qquad$ at LITIS, co-supervised with Eric Sanlaville
		\\ \\	
		{\bf Valentin Pasquale} (L3 ENS Lyon) \hfill {\em summer 2019} 
		\\
		$| \qquad$ \textit{Internship}: Fireworks technique for temporal spanners\\
		$| \qquad$ at LaBRI, co-supervised with Arnaud Casteigts
	\end{rSection}
	
	%\begin{rSection}{Reviews}
	%	
	%	$| \qquad$ Algorithmica \hfill {\em 2022} \\
	%	$| \qquad$ Theoretical Computer Science \\
	%	$| \qquad$ Computing \hfill {\em 2021} \\
	%	$| \qquad$ Computer Networks \\
	%	$| \qquad$ ROADEF \\
	%	$| \qquad$ ALGOSENSORS \\
	%	$| \qquad$ Journal of Interconnection Networks \\
	%	$| \qquad$ Theoretical Computer Science \\
	%	$| \qquad$ Journal of Computer and System Sciences \\
	%	$| \qquad$ Theoretical Computer Science \hfill {\em 2020} \\
	%	$| \qquad$ ALGOSENSORS \\
	%	$| \qquad$ AlgoTel \\
	%	$| \qquad$ Journal of Computer and System Sciences\\
	%	$| \qquad$ International Symposium on Mathematical Foundations of Computer Science \hfill {\em 2019} \\
	%	$| \qquad$ Discrete Applied Mathematics \\
	%	$| \qquad$ CoRes \hfill {\em 2018} 
	%\end{rSection}
	
	\begin{rSection}{Teaching ($\approx$ 350 hours)}
		
%		\textbf{University of Cambridge}\\
%		$| \qquad$ TBD \hfill {\em 2022-2023}
%		\\
		
		\textbf{University of Bordeaux} \\
		$| \qquad$ Mobility algorithms ($2^{nd}$ year Master of Networking)\hfill {\em 2020-2021}\\
		$| \qquad$ Automata theory ($3^{rd}$ year Bachelor of CS)\\
		$| \qquad$ Techniques for algorithms and programming ($3^{rd}$ year Bachelor of CS)\\
		$| \qquad$ Excel and CS basics ($2^{nd}$ year Bachelor of Economics and Management)\\
		$| \qquad$ Array algorithms ($1^{st}$ year Bachelor of Math and CS, \textbf{given in English})\\
		$| \qquad$ CS basics ($1^{st}$ year Bachelor of Math and Science)\\
		$| \qquad$ CS specialty ($1^{st}$ year Bachelor of Math and Science)\\
		$| \qquad$ Mobility algorithms ($2^{nd}$ year Master of Networking)\hfill {\em 2019-2020}\\
		$| \qquad$ Array algorithms ($1^{st}$ year Bachelor of Math and CS)\\
		$| \qquad$ CS basics ($1^{st}$ year Bachelor of Math and Science)\\
		$| \qquad$ Basic data structure algorithms ($2^{nd}$ year Bachelor of CS)\hfill {\em 2018-2019}\\
		$| \qquad$ Networking ($2^{nd}$ year Bachelor of CS)\\
		$| \qquad$ Basic data structure algorithms ($2^{nd}$ year Bachelor of CS)\hfill {\em 2017-2018}\\
		$| \qquad$ Array algorithms ($1^{st}$ year Bachelor of Math and CS)
		\\
		
		\textbf{Bordeaux high schools} \\
		$| \qquad$ MATh.en.JEANS ($\approx$ 14-year-olds)\hfill {\em 2018-2019}\\
		$| \qquad$ Maths à modeler ($\approx$ 17-year-olds)\hfill {\em 2017-2018}
	\end{rSection}
	
	\begin{rSection}{Service} %\hfill \MakeLowercase{\normalfont(lines noted with a \textbf{*} still apply today)}}
		$| \qquad$ ALGOWIN: program committee member \hfill {\em 2024} \\
		$| \qquad$ SAND: program committee member \\
		$| \qquad$ LIPNE complexity seminar: co-organiser \hfill {\em 2023-present}\\
		$| \qquad$ AlgoTel: program committee member \hfill {\em 2022-2024} \\
		$| \qquad$ ANR TEMPOGRAL: member \hfill {\em 2022-present} \\
%		$| \qquad$ AlgoTel: graph session chair \hfill {\em 2021} \\
%		$| \qquad$ Société Informatique de France: member \hfill {\em 2020} \\
%		$| \qquad$ AlgoTel: shadow program committee member \\
		$| \qquad$ IWOCA: organizing committee member \\
		$| \qquad$ LaBRI AlgoDist seminar: co-organiser \hfill {\em 2019-2021}\\
		$| \qquad$ PhD student association Afodib: secretary and seminar organizer \hfill {\em 2018-2021}\\
		%$| \qquad$ MATh.en.JEANS: member \\
		$| \qquad$ FCT: organizing committee member \hfill {\em 2017}\\
		$| \qquad$ $\approx$ 100 reviews for workshops, conferences, and journals \hfill {\em 2017-present}\\
		%$| \qquad$ Maths à modeler: member
	\end{rSection}
%\newpage
	%\begin{rSection}{Summer schools}
	%$| \qquad$ École de Jeunes Chercheurs en Informatique Mathématique \hfill {\em 2020}\\
	%$| \qquad$ ResCom (assistant and attendant)\hfill {\em 2019}
	%\end{rSection}
	%
	%\begin{rSection}{Events otherwise attending/attended}
	%%$| \qquad$ Dagstuhl Seminar Temporal Graphs: Structure, Algorithms, Applications \hfill {\em 2021}\\
	%$| \qquad$ Journée SIF Pandématique (online) \hfill {\em 2020}\\
	%$| \qquad$ Journées Graphes et Algorithmes (online) \\
	%$| \qquad$ GraphMasters workshop (online) \\
	%$| \qquad$ International Workshop on Graph-Theoretic Concepts (online) \\
	%$| \qquad$ ICALP Algorithmic Aspects of Temporal Graphs (online)\\
	%$| \qquad$ French-Israeli Laboratory on Foundations of Computer Science \hfill {\em 2018} \\
	%$| \qquad$ Journées Complexité et Algorithmes \\
	%$| \qquad$ Journées Graphes et Algorithmes \hfill {\em 2017} 
	%\end{rSection}
	%\begin{rSection}{Free time}
	%$| \qquad$ Discovering the world and nature (hiking, bike trips, journeys, ...) \\
	%$| \qquad$ Reading fantasy books or playing games with friends \\
	%$| \qquad$ Occasional jogging or climbing 
	%\end{rSection}
	\begin{center}
		-----------
	\end{center}
	
\end{document}
